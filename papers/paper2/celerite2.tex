% Copyright 2015-2017 Dan Foreman-Mackey and the co-authors listed below.

\documentclass[manuscript, letterpaper]{aastex6}

\pdfoutput=1

%%% This file is generated by the Makefile.
\newcommand{\githash}{108ca20}\newcommand{\gitdate}{2016-09-06}\newcommand{\gitauthor}{Eric Agol}

\usepackage{microtype}
\usepackage{url}
\usepackage{amsmath}
\usepackage{amssymb}
\usepackage{natbib}
\usepackage{multirow}
\bibliographystyle{aasjournal}

% ----------------------------------- %
% start of AASTeX mods by DWH and DFM %
% ----------------------------------- %

% Matrix fix:
% http://tex.stackexchange.com/questions/317824/letter-c-appearing-inside-pmatrix-environment-with-aastex
\makeatletter
\def\env@matrix{\hskip -\arraycolsep % taken from amsmath.sty lines 895ff
  \let\@ifnextchar\new@ifnextchar
  \array{*{\c@MaxMatrixCols}c}}
\makeatother

% Column spacing in matrix
% http://tex.stackexchange.com/questions/275725/adjusting-separation-between-matrix-entries
\setlength\arraycolsep{25pt}

\setlength{\voffset}{0in}
\setlength{\hoffset}{0in}
\setlength{\textwidth}{6in}
\setlength{\textheight}{9in}
\setlength{\headheight}{0ex}
\setlength{\headsep}{\baselinestretch\baselineskip} % this is 2 lines in ``manuscript''
\setlength{\footnotesep}{0in}
\setlength{\topmargin}{-\headsep}
\setlength{\oddsidemargin}{0.25in}
\setlength{\evensidemargin}{0.25in}

\linespread{0.54} % close to 10/13 spacing in ``manuscript''
\setlength{\parindent}{0.54\baselineskip}
\hypersetup{colorlinks = false}
\makeatletter % you know you are living your life wrong when you need to do this
\long\def\frontmatter@title@above{
\vspace*{-\headsep}\vspace*{\headheight}
\noindent\footnotesize
{\noindent\footnotesize\textsc{\@journalinfo}}\par
{\noindent\scriptsize Preprint typeset using \LaTeX\ style AASTeX6 with modifications
}\par\vspace*{-\baselineskip}\vspace*{0.625in}
}%
\makeatother

% Section spacing:
\makeatletter
\let\origsection\section
\renewcommand\section{\@ifstar{\starsection}{\nostarsection}}
\newcommand\nostarsection[1]{\sectionprelude\origsection{#1}}
\newcommand\starsection[1]{\sectionprelude\origsection*{#1}}
\newcommand\sectionprelude{\vspace{1em}}
\let\origsubsection\subsection
\renewcommand\subsection{\@ifstar{\starsubsection}{\nostarsubsection}}
\newcommand\nostarsubsection[1]{\subsectionprelude\origsubsection{#1}}
\newcommand\starsubsection[1]{\subsectionprelude\origsubsection*{#1}}
\newcommand\subsectionprelude{\vspace{1em}}
\makeatother

\sloppy\sloppypar

%\widowpenalty=10000
\clubpenalty=1000000000

% ------------------ %
% end of AASTeX mods %
% ------------------ %

% Projects:
\newcommand{\project}[1]{\textsf{#1}}
\newcommand{\kepler}{\project{Kepler}}
\newcommand{\tess}{\project{TESS}}
\newcommand{\celerite}{\project{celerite}}
\newcommand{\celeriteterm}{\emph{celerite}}
\newcommand{\emcee}{\project{emcee}}

\newcommand{\foreign}[1]{\emph{#1}}
\newcommand{\etal}{\foreign{et\,al.}}
\newcommand{\etc}{\foreign{etc.}}
\newcommand{\ie}{\foreign{i.e.}}

\newcommand{\figureref}[1]{\ref{fig:#1}}
\newcommand{\Figure}[1]{Figure~\figureref{#1}}
\newcommand{\figurelabel}[1]{\label{fig:#1}}

\newcommand{\Table}[1]{Table~\ref{tab:#1}}
\newcommand{\tablelabel}[1]{\label{tab:#1}}

\renewcommand{\eqref}[1]{\ref{eq:#1}}
\newcommand{\Eq}[1]{Equation~(\eqref{#1})}
\newcommand{\eq}[1]{\Eq{#1}}
\newcommand{\eqalt}[1]{Equation~\eqref{#1}}
\newcommand{\eqlabel}[1]{\label{eq:#1}}

\newcommand{\sectionname}{Section}
\newcommand{\sectref}[1]{\ref{sect:#1}}
\newcommand{\Sect}[1]{\sectionname~\sectref{#1}}
\newcommand{\sect}[1]{\Sect{#1}}
\newcommand{\sectalt}[1]{\sectref{#1}}
\newcommand{\App}[1]{Appendix~\sectref{#1}}
\newcommand{\app}[1]{\App{#1}}
\newcommand{\sectlabel}[1]{\label{sect:#1}}

\newcommand{\T}{\ensuremath{\mathrm{T}}}
\newcommand{\dd}{\ensuremath{\,\mathrm{d}}}
\newcommand{\unit}[1]{{\ensuremath{\,\mathrm{#1}}}}
\newcommand{\bvec}[1]{{\ensuremath{\boldsymbol{#1}}}}

% TO DOS
\newcommand{\todo}[3]{{\color{#2}\emph{#1}: #3}}
\newcommand{\dfmtodo}[1]{\todo{DFM}{red}{#1}}
\newcommand{\agoltodo}[1]{\todo{Agol}{blue}{#1}}
\newcommand{\racomment}[1]{{\color{magenta}#1}}


% \shorttitle{}
% \shortauthors{}
% \submitted{Submitted to \textit{The Astrophysical Journal}}

\begin{document}

\title{%
\vspace{-\baselineskip}
Faster Gaussian process modeling
\vspace{-2\baselineskip}
}

\newcounter{affilcounter}
\setcounter{affilcounter}{1}

\edef \sagan {\arabic{affilcounter}}\stepcounter{affilcounter}
\altaffiltext{\sagan}{Sagan Fellow}

\edef \uw {\arabic{affilcounter}}\stepcounter{affilcounter}
\altaffiltext{\uw}{Astronomy Department, University of Washington,
                   Seattle, WA}

\edef \iis {\arabic{affilcounter}}\stepcounter{affilcounter}
\altaffiltext{\iis}{Department of Computational and Data Sciences,
                    Indian Institute of Science, Bangalore, India}


\author{%
    Daniel~Foreman-Mackey\altaffilmark{\sagan,\uw},
    Sivaram~Ambikasaran\altaffilmark{\iis},
    Eric~Agol\altaffilmark{\uw}, and
    TBD
}



\begin{abstract}

    Much faster solves using Cholesky solver.

\end{abstract}

\keywords{%
 methods: data analysis
 ---
 methods: statistical
 ---
 asteroseismology
 ---
 stars: rotation
 ---
 planetary systems
}

\section{Introduction}

Gaussian Processes \citep[GPs;][]{Rasmussen:2006}. We made this faster.

\section{Summary}

This software is available on GitHub
\url{https://github.com/dfm/celerite}\footnote{This version of the paper was
generated with git commit \texttt{\githash} (\gitdate).} and Zenodo
\dfmtodo{add link}, and it is made available under the MIT license.

\acknowledgments
It is a pleasure to thank
\ldots
for helpful discussions informing the ideas and code presented here.

This work was performed in part under contract with the Jet Propulsion
Laboratory (JPL) funded by NASA through the Sagan Fellowship Program executed
by the NASA Exoplanet Science Institute.
EA acknowledges support from NASA grants NNX13AF20G, NNX13A124G, NNX13AF62G,
from National Science Foundation (NSF) grant AST-1615315, and from
NASA Astrobiology Institute's Virtual Planetary Laboratory, supported
by NASA under cooperative agreement NNH05ZDA001C.

This research made use of the NASA \project{Astrophysics Data System} and the
NASA Exoplanet Archive.
The Exoplanet Archive is operated by the California Institute of Technology,
under contract with NASA under the Exoplanet Exploration Program.

This paper includes data collected by the \kepler\ mission. Funding for the
\kepler\ mission is provided by the NASA Science Mission directorate.
We are grateful to the entire \kepler\ team, past and present.
These data were obtained from the Mikulski Archive for Space Telescopes
(MAST).
STScI is operated by the Association of Universities for Research in
Astronomy, Inc., under NASA contract NAS5-26555.
Support for MAST is provided by the NASA Office of Space Science via grant
NNX13AC07G and by other grants and contracts.

This research made use of Astropy, a community-developed core Python package
for Astronomy \citep{Astropy-Collaboration:2013}.

\facility{Kepler}
\software{%
     \project{AtroPy} \citep{Astropy-Collaboration:2013},
     \project{corner.py} \citep{Foreman-Mackey:2016},
     \project{Eigen} \citep{Guennebaud:2010},
     \project{emcee} \citep{Foreman-Mackey:2013},
     \project{george} \citep{Ambikasaran:2016},
     \project{LAPACK} \citep{Anderson:1999},
     \project{matplotlib} \citep{Hunter:2007},
     \project{numpy} \citep{Van-Der-Walt:2011},
     \project{transit} \citep{Foreman-Mackey:2016a},
     \project{scipy} \citep{Jones:2001}.
}

\bibliography{celerite2}

\end{document}
